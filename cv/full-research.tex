\documentclass[]{article}
\usepackage[T1]{fontenc}
\usepackage{lmodern}
\usepackage{amssymb,amsmath}
\usepackage{ifxetex,ifluatex}
\usepackage{fixltx2e} % provides \textsubscript
% use microtype if available
\IfFileExists{microtype.sty}{\usepackage{microtype}}{}
\ifnum 0\ifxetex 1\fi\ifluatex 1\fi=0 % if pdftex
  \usepackage[utf8]{inputenc}
\else % if luatex or xelatex
  \usepackage{fontspec}
  \ifxetex
    \usepackage{xltxtra,xunicode}
  \fi
  \defaultfontfeatures{Mapping=tex-text,Scale=MatchLowercase}
  \newcommand{\euro}{€}
\fi
% Redefine labelwidth for lists; otherwise, the enumerate package will cause
% markers to extend beyond the left margin.
\makeatletter\AtBeginDocument{%
  \renewcommand{\@listi}
    {\setlength{\labelwidth}{4em}}
}\makeatother
\usepackage{enumerate}
\ifxetex
  \usepackage[setpagesize=false, % page size defined by xetex
              unicode=false, % unicode breaks when used with xetex
              xetex]{hyperref}
\else
  \usepackage[unicode=true]{hyperref}
\fi
\hypersetup{breaklinks=true,
            bookmarks=true,
            pdfauthor={},
            pdftitle={},
            colorlinks=true,
            urlcolor=blue,
            linkcolor=magenta,
            pdfborder={0 0 0}}
\setlength{\parindent}{0pt}
\setlength{\parskip}{6pt plus 2pt minus 1pt}
\setlength{\emergencystretch}{3em}  % prevent overfull lines
\setcounter{secnumdepth}{0}

\author{}
\date{}

\begin{document}

\section{Research Statement}

\subsection{Motivation}

The goal of my research is to enhance the reliability, maintainability,
and security of software systems through programming language
technology. The design of programming languages strongly determines the
properties of programs written in them. A good language will promote the
design of trustworthy and robust software and discourage the creation of
insecure and inflexible code.

In pursuit of this goal, my research focuses on the design of
\emph{statically-typed programming languages.} Static type systems are a
popular, cost-effective form of lightweight program verification. They
provide a tractable and modular way for programmers to express
properties that can be mechanically checked by the compiler, thereby
ruling out a wide variety of common memory manipulation and control-flow
errors. For example, systems written with type-safe languages cannot be
compromised by buffer overruns if all array accesses are statically
proven safe. In general, the value of a static type system rests in its
ability to express the invariants and consistency properties of software
systems.

However, the static type systems of modern programming languages are
limited in their expressiveness. Some programming idioms must be ruled
out simply because they cannot be shown to be sound by a particular type
system. My research explores novel methods to bring more expressive
types systems to users: enhancing the capabilities of existing type
systems to reason about run-time structure and behavior, reconciling the
features of expressive-but-theoretical type systems with existing
languages, partially automating the process of type system design so
that we may be more confident in the soundness of more complicated type
systems, and incorporating programming logics into the design of
practical type systems so that application-specific properties may be
expressed.

Below, I describe this research in more detail as well as my future
directions.

\subsection{Semantics and Design of Dependently-typed Languages}

While extremely helpful for building robust software, the type systems
used in practice verify only relatively weak safety properties; they
fall far short of what is needed to do full verification of program
correctness. This inexpressiveness is partly by design---full program
verification is potentially expensive, not fully automatable, and
probably unwarranted for non-safety-critical software. Nevertheless,
there are many situations in which the ability to specify program
properties richer than current type systems permit would be useful. In
fact, there is a spectrum of possibilities between simple type safety
and full program verification.

Dependent type systems extend the type systems of current practice to
allow incremental, lightweight specifications. They work by allowing
types that are used as specifications to depend on values computed by
program expressions. However, even though the theory behind dependent
type systems has been refined over several decades, such type systems
have been little used in practical programming languages.

The \textbf{Trellys project}, a joint project between my group at Penn,
and groups at the University of Iowa and Portland State University,
investigated the design and implementation of a programming language
with dependent types, called Trellys. Trellys is the only research group
addressing issues in the design of a call-by-value, full spectrum
dependently-typed language.

\begin{enumerate}[1.]
\item
  An investigation of the semantics of equality in a call-by-value
  language with nontermination and dependent-types (Casinghino, Sjöberg,
  and Weirich 2014,,,).
\item
  Practical considerations for a dependently-typed language, including
  type inference {[}@congruence{]} and computational irrelevance
  (Sjöberg et al. 2012).
\end{enumerate}

I've also work to build the community of researchers that study the
interaction between dependent types and functional programming. In
particular, I organized two meetings: the DTP workshop co-located with
ICFP 2013, and the Shonan Seminar 007 in 2011.

This work has been recognized through invitations for keynote addresses
(PLMW 2014, RTA 2011, MFPS 2011), lectures at the OPLSS summer school
(2014 and 2013), and discussion sessions at the PLPV workshop (2011 and
2010).

\subsection{Extending the Haskell Type System}

The Haskell programming language has been dramatically growing in
popularity in the past five years. Part of this growth can be attributed
to its ``incredibly expressive static type system.''

Haskell has one of the world's most advanced type systems. Because of
the advantage that this type system brings

For the past ten years, I have been working with Simon Peyton Jones,
Dimitrios Vytiniotis and many others on extensions to the type system of
the Glasgow Haskell Compiler. In that space, I've contributed to several
extensions to both the source and core languages, and Roles.

\begin{enumerate}[1.]
\item
  Language extensions for dependently-typed programming in GHC
\end{enumerate}

Generalized Algebraic Datatypes (S. L. P. Jones et al. 2006), Datatype
Promotion (Weirich, Hsu, and Eisenberg 2013,), Type families(Eisenberg
et al. 2014).

\begin{enumerate}[1.]
\setcounter{enumi}{1}
\item
  Haskell libraries The singletons library (used by the units library
  for physical simulation). The RepLib library (used by the Unbound
  library for modeling variable binding).
\item
  Modularity and notions of type equality {[}@coercions,@roles{]}
\item
  First-class polymorphism {[}{]}
\end{enumerate}

This work has been recognized through invitations for keynote addresses
(ICFP 2014, FLOPS 2012) and chairing of ICFP 2010 and Haskell Symposium
2009.

Member of IFIP WG 2.8

** Mechanizing programming language metatheory

Because program security guarantees are often based on static type
checking, confidence in the soundness of static type systems is
essential. Yet, as static type systems become more and more expressive,
such meta-theoretic results become more complex. Typically proofs about
the properties of programming languages are done by hand, despite the
length and complexity of these results for modern languages. These
proofs are not difficult---they use standard, well-understood
techniques---but they are often overwhelming in the details.

Therefore, I have been working to make the use of automated proof
assistants more commonplace in the formalization of programming language
metatheory. This project has two components: developing community
infrastructure (education materials, workshops, libraries, electronic
fora) to get researchers to use existing tools, and developing new
technologies for programming language representation, specifically, the
treatment of binding constructs, to make this process easier.

I have been working on this project in collaboration with Benjamin
Pierce and Steve Zdancewic at the University of Pennsylvania as well as
Peter Sewell at Cambridge University and Randy Pollack at the University
of Edinburgh and a number of Penn students. This project was supported
by the NSF CISE Computing Research Infrastructure program and the DARPA
Computer Science Study Group.

\emph{*} POPLmark challenge and educational infrastructure

In 2005, the POPLmark team issued the POPLmark challenge: a set of
design problems to both assess and advance the current best practices in
machine-assisted support for the formalization of programming languages.
This challenge was met with enthusiasm and spurred vigorous debate.
Benjamin Pierce and I co-edited a \href{jar}{journal special issue}
devoted to descriptions of the POPLmark challenge. I was invited to give
retrospective talks at Cambridge University (Dec 2009) and LFMTP 2012.

Based on this initial success, I organized and chaired the 2006 Workshop
on Mechanizing Metatheory, which was held in conjunction with ICFP, and
attended by over fifty participants. This workshop had additional
meetings in 2007, 2008, 2009, and 2010. Furthermore, we organized a
tutorial on using Coq for programming language metatheory at POPL 2008
and I repeated this tutorial at OPLSS 2008.

\emph{*} Binding representations and tools

LNgen,

\begin{itemize}
\item
  Future plans
\end{itemize}

\subsection{TRELLYS}

The TRELLYS project seeks to design a dependently-typed programming
language that supports both programming and theorem proving in the same
framework. With \href{http://homepage.cs.uiowa.edu/\%7Eastump/}{Aaron
Stump} (Iowa State) and \href{http://web.cecs.pdx.edu/\%7Esheard/}{Tim
Sheard} (Portland State), we have been discovering what it means for
programs and proofs to interact. See our recent
\href{papers/modal.pdf}{paper submission}, as well as papers in
\href{http://www.seas.upenn.edu/\%7Esweirich/papers/msfp12log.pdf}{MSFP}
\href{http://www.seas.upenn.edu/\%7Esweirich/papers/msfp12prog.pdf}{2012}
and
\href{http://www.seas.upenn.edu/\%7Esweirich/papers/plpv2012genreccbv.pdf}{PLPV
2012}. Also, slides from \href{talks/TrellysPLPV.pdf}{PLPV 2010} and
\href{talks/tlca-2011.pdf}{RDP 2011}, give an overview of our work.

\subsection{Dependently-Typed Haskell}

Working with
\href{http://research.microsoft.com/en-us/people/simonpj/}{Simon Peyton
Jones} and
\href{http://research.microsoft.com/en-us/people/dimitris/}{Dimitrios
Vytiniotis} at MSR Cambridge, Penn students Brent Yorgey, Richard
Eisenberg and Justin Hsu and I have been working to extend the type
system of Haskell with better support for dependently-typed programing.
\\ - \href{papers/nokinds-extended.pdf}{Down with kinds: adding
dependent heterogenous equality to FC} (in submission) extends GHC's
core language with support for kind equalities.\\-
\href{http://www.cis.upenn.edu/\%7Eeir/papers/2012/singletons/paper.pdf}{Dependently-typed
programming with singletons} (Haskell Symposium 2012) discusses the
\href{http://hackage.haskell.org/package/singletons}{singletons} library
that automatically generates infrastructure for dependently-typed
programming.\\- \href{papers/tldi12.pdf}{Giving Haskell a Promotion}
(TLDI 2012, with \href{http://dreixel.net/}{Jos Pedro Maglhaes} and
\href{http://gallium.inria.fr/\%7Ejcretin/}{Julien Cretin}) adds kind
polymorphism and datatype promotion to GHC.

\subsection{Type-directed programming}

Defining functions via type information cuts down on boilerplate
programming as many operations may be defined once, for all types of
data. With my student \href{http://www.cis.upenn.edu/\%7Ebyorgey/}{Brent
Yorgey} and collaborator \href{http://web.cecs.pdx.edu/\%7Esheard/}{Tim
Sheard} (Portland State), I developed
\href{http://hackage.haskell.org/package/unbound/}{Unbound}, a
\href{http://www.haskell.org/ghc/}{Haskell} library for declaratively
specifying binding structure and automatically generating free variable,
alpha-equivalence and substitution functions. This library is built
using \href{http://hackage.haskell.org/package/RepLib/}{RepLib}, an
expressive library that I developed for generic programming in Haskell.
My student \href{http://www.seas.upenn.edu/\%7Eccasin/}{Chris
Casinghino}and I explored
\href{http://www.seas.upenn.edu/\%7Esweirich/papers/aritygen.pdf}{arity
and type generic programming in Agda},
\href{http://www.seas.upenn.edu/\%7Esweirich/ssgip/}{lecture notes} from
the\href{http://www.comlab.ox.ac.uk/projects/gip/school.html}{Spring
School on Generic and Indexed Programming} provide a gentle
introduction.\\

\subsection{Machine assistance for programming languages research}

Designing and proving properties about programming languages is hard,
but the proofs themselves are straightforward once you know how to set
them up. At the same time, it is all too easy to miss the one little
case that ruins the whole ``proof''. Modern proof assistants, such as
\href{http://www.cs.cmu.edu/\%7Etwelf/}{Twelf},
\href{http://coq.inria.fr/}{Coq}, and
\href{http://www.cl.cam.ac.uk/Research/HVG/Isabelle/}{Isabelle} are good
at expressing this sort of reasoning, but it is hard to know where to
start. I've been working with the
\href{http://www.cis.upenn.edu/proj/plclub/mmm/}{POPLmark team} to issue
\href{http://www.cis.upenn.edu/proj/plclub/mmm/}{challenge problems},
organize a \href{http://www.seas.upenn.edu/\%7Esweirich/wmm/}{workshop},
explore
\href{http://www.seas.upenn.edu/\%7Esweirich/papers/nominal-coq/}{techniques
for reasoning about binding}, and develop
\href{http://www.seas.upenn.edu/\%7Esweirich/cis700/f06/}{educational
materials} about mechanizing programming language metatheory. Brian
Aydemir developed a
\href{http://www.cis.upenn.edu/\%7Eplclub/metalib/}{library} for
programming language metatheory (used in the Penn tutorials) and
\href{http://www.cis.upenn.edu/\%7Ebaydemir/papers/lngen/}{LNgen}, a
tool for automatically proving properties about binding.\\ Type
inference for advanced type systems
----------------------------------------

Advanced type system features, such as
\href{papers/putting.pdf}{first-class polymorphism},
\href{http://www.seas.upenn.edu/\%7Esweirich/papers/icfp08.pdf}{impredicative
polymorphism} and
\href{http://www.seas.upenn.edu/\%7Esweirich/papers/gadt.pdf}{generalized
algebraic datatypes}, do not interact well with the standard algorithms
for type inference in modern typed functional languages, such as ML and
Haskell.~ \href{http://research.microsoft.com/\%7Esimonpj/}{Simon Peyton
Jones}, my students Dimitrios Vytiniotis, Geoffrey Washburn and I have
incorporated ideas from Local Type Inference to extend the
\href{http://www.haskell.org/ghc/}{Glasgow Haskell Compiler} with
support for these features. \\ Combinatorial Species
---------------------

\section{References}

Casinghino, Chris, Vilhelm Sjöberg, and Stephanie Weirich. 2014.
``Combining Proofs and Programs in a Dependently Typed Language.'' In
\emph{POPL 2014: 41st ACM SIGPLAN-SIGACT Symposium on Principles of
Programming Languages}, 33--45. San Diego, CA, USA.

Eisenberg, Richard A., Dimitrios Vytiniotis, Simon Peyton Jones, and
Stephanie Weirich. 2014. ``Closed type families with overlapping
equations.'' In \emph{POPL 2014: 41st ACM SIGPLAN-SIGACT Symposium on
Principles of Programming Languages}, 671--683. San Diego, CA, USA.

Jia, Limin, Jianzhou Zhao, Vilhem Sjöberg, and Stephanie Weirich. 2010.
``Dependent types and Program Equivalence.'' In \emph{37th ACM
SIGACT-SIGPLAN Symposium on Principles of Programming Languages (POPL)},
275--286. Madrid, Spain.

Jones, Simon L. Peyton, Dimitrios Vytiniotis, Stephanie Weirich, and
Geoffrey Washburn. 2006. ``Simple unification-based type inference for
GADTs.'' In \emph{International Conference on Functional Programming
(ICFP)}, 50--61. Portland, OR, USA.

Kimmel, Garrin, Aaron Stump, Harley D. Eades, Peng Fu, Tim Sheard,
Stephanie Weirich, Chris Casinghino, Vilhelm Sjöberg, Nathin Collins,
and Ki Yunh Anh. 2013. ``Equational reasoning about programs with
general recursion and call-by-value semantics.'' \emph{Progress in
Informatics} (mar): 19--48.
\href{http://www.nii.ac.jp/pi/}{http://www.nii.ac.jp/pi/}.

Sjöberg, Vilhelm, Chris Casinghino, Ki Yung Ahn, Nathan Collins, Harley
D. Eades III, Peng Fu, Garrin Kimmell, Tim Sheard, Aaron Stump, and
Stephanie Weirich. 2012. ``Irrelevance, Heterogenous Equality, and
Call-by-value Dependent Type Systems.'' In \emph{Fourth workshop on
Mathematically Structured Functional Programming (MSFP '12)}, 112--162.

Stump, Aaron, Vilhelm Sjöberg, and Stephanie Weirich. 2010.
``Termination Casts: A Flexible Approach to Termination with General
Recursion.'' In \emph{Workshop on Partiality and Recursion in
Interactive Theorem Provers}, 76--93. Edinburgh, Scotland.

Weirich, Stephanie, Justin Hsu, and Richard A. Eisenberg. 2013. ``System
FC with explicit kind equality.'' In \emph{Proceedings of The 18th ACM
SIGPLAN International Conference on Functional Programming}, 275--286.
Boston, MA.

Yorgey, Brent A., Stephanie Weirich, Julien Cretin, Simon Peyton Jones,
Dimitrios Vytiniotis, and José Pedro Magalha\textbackslash{} es. 2012.
``Giving Haskell A Promotion.'' In \emph{Seventh ACM SIGPLAN Workshop on
Types in Language Design and Implementation (TLDI '12)}, 53--66.

\end{document}
